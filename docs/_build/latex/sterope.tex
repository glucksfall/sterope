%% Generated by Sphinx.
\def\sphinxdocclass{report}
\documentclass[letterpaper,10pt,english]{sphinxmanual}
\ifdefined\pdfpxdimen
   \let\sphinxpxdimen\pdfpxdimen\else\newdimen\sphinxpxdimen
\fi \sphinxpxdimen=.75bp\relax

\PassOptionsToPackage{warn}{textcomp}
\usepackage[utf8]{inputenc}
\ifdefined\DeclareUnicodeCharacter
% support both utf8 and utf8x syntaxes
\edef\sphinxdqmaybe{\ifdefined\DeclareUnicodeCharacterAsOptional\string"\fi}
  \DeclareUnicodeCharacter{\sphinxdqmaybe00A0}{\nobreakspace}
  \DeclareUnicodeCharacter{\sphinxdqmaybe2500}{\sphinxunichar{2500}}
  \DeclareUnicodeCharacter{\sphinxdqmaybe2502}{\sphinxunichar{2502}}
  \DeclareUnicodeCharacter{\sphinxdqmaybe2514}{\sphinxunichar{2514}}
  \DeclareUnicodeCharacter{\sphinxdqmaybe251C}{\sphinxunichar{251C}}
  \DeclareUnicodeCharacter{\sphinxdqmaybe2572}{\textbackslash}
\fi
\usepackage{cmap}
\usepackage[T1]{fontenc}
\usepackage{amsmath,amssymb,amstext}
\usepackage{babel}
\usepackage{times}
\usepackage[Bjarne]{fncychap}
\usepackage{sphinx}

\fvset{fontsize=\small}
\usepackage{geometry}

% Include hyperref last.
\usepackage{hyperref}
% Fix anchor placement for figures with captions.
\usepackage{hypcap}% it must be loaded after hyperref.
% Set up styles of URL: it should be placed after hyperref.
\urlstyle{same}

\addto\captionsenglish{\renewcommand{\figurename}{Fig.\@ }}
\makeatletter
\def\fnum@figure{\figurename\thefigure{}}
\makeatother
\addto\captionsenglish{\renewcommand{\tablename}{Table }}
\makeatletter
\def\fnum@table{\tablename\thetable{}}
\makeatother
\addto\captionsenglish{\renewcommand{\literalblockname}{Listing}}

\addto\captionsenglish{\renewcommand{\literalblockcontinuedname}{continued from previous page}}
\addto\captionsenglish{\renewcommand{\literalblockcontinuesname}{continues on next page}}
\addto\captionsenglish{\renewcommand{\sphinxnonalphabeticalgroupname}{Non-alphabetical}}
\addto\captionsenglish{\renewcommand{\sphinxsymbolsname}{Symbols}}
\addto\captionsenglish{\renewcommand{\sphinxnumbersname}{Numbers}}

\addto\extrasenglish{\def\pageautorefname{page}}

\setcounter{tocdepth}{2}



\title{sterope Documentation}
\date{Feb 28, 2019}
\release{}
\author{Rodrigo Santibáñez}
\newcommand{\sphinxlogo}{\vbox{}}
\renewcommand{\releasename}{}
\makeindex
\begin{document}

\pagestyle{empty}
\sphinxmaketitle
\pagestyle{plain}
\sphinxtableofcontents
\pagestyle{normal}
\phantomsection\label{\detokenize{index::doc}}


Sterope is a python3 package that implement a method based on the Dynamic
Influence Network (\sphinxurl{https://www.ncbi.nlm.nih.gov/pubmed/28866584}) to analyze the
sensitivity of parameter values in the response of a Rule-Based Model written in
kappa (\sphinxurl{https://kappalanguage.org/})

Sterope creates models samples and analyze the Dynamic Influence Network employing
the Sobol method included in the SALib library (\sphinxhref{https://joss.theoj.org/papers/431262803744581c1d4b6a95892d3343}{SALibpaper}). After samples are
created, Sterope simulates them in parallel employing SLURM (\sphinxhref{https://slurm.schedmd.com/}{SLURM}) or the
python multiprocessing API.

The plan to add methods into Pleiades (\sphinxurl{https://github.com/glucksfall/pleiades})
includes a parameterization employing a Particle Swarm Optimization protocol and
other analysis methods that are typical of frameworks like Ordinary Differential
Equations. You could write us if you wish to add methods into pleione or aid in
the development of them.


\chapter{Installation}
\label{\detokenize{Installation:installation}}\label{\detokenize{Installation::doc}}
There are two different ways to install sterope:
\begin{enumerate}
\def\theenumi{\arabic{enumi}}
\def\labelenumi{\theenumi .}
\makeatletter\def\p@enumii{\p@enumi \theenumi .}\makeatother
\item {} 
\sphinxstylestrong{Install sterope natively (Recommended).}

\sphinxstyleemphasis{OR}

\item {} 
\sphinxstylestrong{Clone the Github repository.} If you are familiar with git, sterope can
be cloned and the respective folder added to the python path. Further details
are below.

\end{enumerate}

\begin{sphinxadmonition}{note}{Note:}
\sphinxstylestrong{Need Help?}
If you run into any problems with installation, please visit our chat room:
\sphinxurl{https://gitter.im/glucksfall/pleiades}
\end{sphinxadmonition}


\section{Option 1: Install sterope natively on your computer}
\label{\detokenize{Installation:option-1-install-sterope-natively-on-your-computer}}
The recommended approach is to use system tools, or install them if
necessary. To install python packages, you could use pip, or download
the package from \sphinxhref{https://test.pypi.org/project/sterope/}{python package index}.
\begin{enumerate}
\def\theenumi{\arabic{enumi}}
\def\labelenumi{\theenumi .}
\makeatletter\def\p@enumii{\p@enumi \theenumi .}\makeatother
\item {} 
\sphinxstylestrong{Install with system tools}

With pip, you simple need to execute and sterope will be installed on
\sphinxcode{\sphinxupquote{\$HOME/.local/lib/python3.6/site-packages}} folder or similar.

\begin{sphinxVerbatim}[commandchars=\\\{\}]
pip3 install \PYGZhy{}i https://test.pypi.org/simple/ sterope \PYGZhy{}\PYGZhy{}user
\end{sphinxVerbatim}

If you have system rights, you could install sterope for all users with

\begin{sphinxVerbatim}[commandchars=\\\{\}]
sudo \PYGZhy{}H pip3 install \PYGZhy{}i https://test.pypi.org/simple/ sterope
\end{sphinxVerbatim}

\item {} 
\sphinxstylestrong{Download from python package index}

Alternatively, you could download the package (useful when pip fails to
download the package because of lack of SSL libraries) and then install with pip.
For instance:

\begin{sphinxVerbatim}[commandchars=\\\{\}]
wget https://test\PYGZhy{}files.pythonhosted.org/packages/7a/d5/bd1f28031d9be331cbd5a7945a2934536c8ccf7c7171a80b4bde132ee245/sterope\PYGZhy{}1.0.1\PYGZhy{}py3\PYGZhy{}none\PYGZhy{}any.whl
pip3 install sterope\PYGZhy{}1.0.1\PYGZhy{}py3\PYGZhy{}none\PYGZhy{}any.whl \PYGZhy{}\PYGZhy{}user
\end{sphinxVerbatim}

\begin{sphinxadmonition}{note}{Note:}
\sphinxstylestrong{Why Python3?}:
Sterope is intended to be used with python3, because python2 won’t receive
further development past Jan 1st, 2020. Although, the code has specific python3
functions over dictionaries and f-strings.
\end{sphinxadmonition}

\begin{sphinxadmonition}{note}{Note:}
\sphinxstylestrong{pip, Python and Anaconda}:
Be aware which pip you invoque. You could install pip3 with
\sphinxcode{\sphinxupquote{sudo apt-get install python3-pip}} if you have system rights, or
install python3 from source, and adding \sphinxcode{\sphinxupquote{\textless{}python3 path\textgreater{}/bin/pip3}} to the
path, or linking it in a directory like \sphinxcode{\sphinxupquote{\$HOME/bin}} which is commonly
added to the path at system login. Please be aware that, if you installed
Anaconda, pip could be linked to the Anaconda specific version of pip, which
will install sterope into Anaconda’s installation folder.
Type \sphinxcode{\sphinxupquote{which pip3}} to find out the source of pip, and type \sphinxcode{\sphinxupquote{python3 -m site}}
to find out where is more likely sterope would be installed.
\end{sphinxadmonition}

\end{enumerate}


\section{Option 2: Clone the Github repository}
\label{\detokenize{Installation:option-2-clone-the-github-repository}}\begin{enumerate}
\def\theenumi{\arabic{enumi}}
\def\labelenumi{\theenumi .}
\makeatletter\def\p@enumii{\p@enumi \theenumi .}\makeatother
\item {} 
\sphinxstylestrong{Clone with git}

The source code is uploaded and maintained through Github at
\sphinxurl{https://github.com/glucksfall/sterope}. Therefore, you could clone the
repository locally, and then add the folder to the \sphinxcode{\sphinxupquote{PYTHONPATH}}. Beware
that you should install the \sphinxstyleemphasis{pandas} (\sphinxhref{https://pandas.pydata.org/}{pandas}), \sphinxstyleemphasis{seaborn} (\sphinxhref{https://seaborn.pydata.org/}{seaborn}), and
\sphinxstyleemphasis{SALib} (\sphinxhref{https://salib.readthedocs.io/en/latest/}{SALib}) packages by any means.

\begin{sphinxVerbatim}[commandchars=\\\{\}]
git clone https://github.com/glucksfall/sterope /opt
\PYG{n+nb}{echo} \PYG{n+nb}{export} \PYG{n+nv}{PYTHONPATH}\PYG{o}{=}\PYG{l+s+s2}{\PYGZdq{}\PYGZbs{}\PYGZdl{}PYTHONPATH:/opt/sterope\PYGZdq{}} \PYGZgt{}\PYGZgt{} \PYG{n+nv}{\PYGZdl{}HOME}/.profile
\end{sphinxVerbatim}

\begin{sphinxadmonition}{note}{Note:}
Adding the path to \sphinxcode{\sphinxupquote{\$HOME/.profile}} allows python to find the package
installation folder after each user login. Similarly, adding the path to
\sphinxcode{\sphinxupquote{\$HOME/.bashrc}} allows python to find the package after each terminal
invocation.
\end{sphinxadmonition}

\end{enumerate}


\chapter{Indices and tables}
\label{\detokenize{index:indices-and-tables}}\begin{itemize}
\item {} 
\DUrole{xref,std,std-ref}{genindex}

\item {} 
\DUrole{xref,std,std-ref}{modindex}

\item {} 
\DUrole{xref,std,std-ref}{search}

\end{itemize}



\renewcommand{\indexname}{Index}
\printindex
\end{document}